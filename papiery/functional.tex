\documentclass{article}

\usepackage[utf8]{inputenc}
\usepackage{polski}
\usepackage{titlesec}

\title{Generator ruchu Google Analytics \\ Iteracja I specyfikacja wymagań}
\author{Bartłomiej Dalak \and Bartłomiej Karwowski \and Bartosz Gromek \and Tomasz Kanas}

\begin{document}
\maketitle

\section{Opis ogólny}

Celem tej iteracji jest stworzenie aplikacji generującej statystyki odwiedzania danej strony w danej chwili w Google Analytics.

\section{Wymagania funkcjonalne}

Aplikacja ma umożliwić użytkownikowi przesłanie do Google Analytics statystyk dotyczących odwiedzeń danej strony. Konieczne do tego jest posiadanie przez użytkownika poprawnie skonfigurowanego konta w Google Analytics. Aplikacja ma za zadanie wygenerować w Google Analytics statystyki przedstawiające zdefiniowaną przez użytkownika ilość odwiedzeń strony w przeciągu krótkiego przedziału czasu. Statystyki mają obejmować przeglądarkę odwiedzających i ich lokalizację (w obrębie Polski). Aplikacja będzie całkowicie konsolowa --- nie będzie posiadać graficznego interfejsu. Jedynymi parametrami możliwymi do wyspecyfikowania przez użytkownika (i jednocześnie wymaganymi) są:
\begin{itemize}
\item Tracking ID --- identyfikator konta w Google Analytics
\item URL --- adres strony
\item ilość odwiedzeń
\item wielkość przedziału czasu (w ms), mniejsza niż 4 godziny
\end{itemize}

\subsection{Jakość}
Generowane przez aplikację statystyki dotyczące lokalizacji i przeglądarek odwiedzających są realistyczne, czyli lokalizacja jest rozłożona po całej Polsce w korelacji z gęstością zamieszkania, natomiast ilość odwiedzeń z konkretnych przeglądarek jest skorelowana z ich popularnością w Polsce. Ilość odwiedzeń rozkłada się równo na całym przedziale czasu. Ponadto statystyki generowane są z pewną wariancją.

\subsection{Przykład użycia}
Jedyny możliwy sposób użycia aplikacji. Użytkownik uruchamia program z linii poleceń, przekazując mu wymagane argumenty. Po skończeniu działania programu wszystkie oczekiwane statystyki zostały poprawnie wysłane do Google Analytics.

\subsubsection{Aktorzy}
\begin{itemize}
\item Użytkownik
\item Google Analytics
\end{itemize}

\subsubsection{Dane wejściowe}
\begin{itemize}
\item Tracking ID
\item URL
\item liczba odwiedzeń (z góry ustalonego przedziału)
\item wielkość przedziału czasu (w ms)
\end{itemize}

\subsubsection{Warunki wstępne}
\begin{itemize}
\item Działające połączenie z internetem
\item Posiadanie strony internetowej
\item Posiadanie poprawnie skonfigurowanego konta w Google Analytics
\end{itemize}

\subsubsection{Warunki końcowe}
Do serwera Google Analytics zostało wysłane dokładnie tyle odwiedzeń ile zostało przekazane jako parametr.

\subsubsection{Rezultat}
Użytkownik posiada na swoim koncie Google Analytics tyle dodatkowych odwiedzeń ile chciał, uwzględniających lokalizację i przeglądarkę odwiedzających.

\subsubsection{Scenariusz główny}
\begin{enumerate}
\item Użytkownik uruchamia program z linii poleceń przekazując mu odpowiednie argumenty
\item Program weryfikuje poprawność podanych argumentów
\item Jeśli argumenty były poprawne program wysyła dane do Google Analytics, wypisuje informację o powodzeniu i kończy działanie. W przeciwnym przypadku program wypisuje odpowiedni komunikat o błędzie i kończy działanie.
\end{enumerate}

\subsubsection{Scenariusz alternatywny}
Jeśli w trakcie wysyłania danych wystąpiłaby utrata połączenia z internetem, program informuje o tym użytkownika i oczekuje na powrót połączenia. Użytkownik może ręcznie zakończyć pracę programu, wtedy część danych nie zostanie wysłana.

\subsubsection{Scenariusz neagtywny}
\begin{enumerate}
\item Użytkownik jako argument podaje niepoprawną liczbę odwiedzeń (spoza zadanego przedziału). W takiej sytuacji aplikacja wypisuje na standardowe wyjście odpowiedni komunikat i oczekuje nowych danych.

\item Użytkownik podaje poprawne argumenty, jednak liczba odwiedzeń w stosunku do wielkości przedziału czasu jest na tyle duża, że aplikacja nie jest w stanie wygenerować wiarygodnych danych, generujących oczekiwane statystyki. Wiarygodność danych jest określana na podstawie przeciętnego ruchu w sieci na terenie Polski. W takiej sytuacj aplikacja wypisuje na standardowe wyjście odpowiednie ostrzeżenie, z pytaniem czy użytkownik chce zmodyfikować argumenty.
\end{enumerate}

\section{Wymagania niefunkcjonalne}

\subsection{Niezawodność}
Aplikacja jest zabezpieczona przed utratą połączenia z internetem --- wstrzymuje działanie do momentu odzyskania go bądź otrzymania od użytkownika sygnału o zakończeniu pracy. W takiej sytuacji użytkownik zobaczy komunikat, że wygenerowane statystyki mogą być niepełne i nie spełniać jego oczekiwań.

\subsection{Wydajność}
Aplikacja jest w stanie generować kilka tysięcy odwiedzeń w przeciągu kilku sekund.

\subsection{Bezpieczeństwo}
Aplikacja nie zapisuje żadnych danych użytkownika i wszystkie dane przesyłane są poprzez szyfrowane połączenie (HTTPS).

\end{document}
