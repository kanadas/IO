\documentclass{article}

\usepackage[utf8]{inputenc}
\usepackage{polski}
\usepackage{titlesec}

\title{Generator ruchu Google Analytics \\ Iteracja I specyfikacja wymagań}
\author{Bartłomiej Dalak \and Bartłomiej Karwowski \and Bartosz Gromek \and Tomasz Kanas}

\begin{document}
\maketitle

\section{Opis ogólny}

Celem tej iteracji jest stworzenie aplikacji generującej statystyki odwiedzania danej strony w podanym przez użytownika przedziale czasu w Google Analytics.

\section{Wymagania funkcjonalne}

Aplikacja ma umożliwić użytkownikowi przesłanie do Google Analytics statystyk dotyczących odwiedzeń danej strony. Konieczne do tego jest posiadanie przez użytkownika poprawnie skonfigurowanego konta w Google Analytics. Aplikacja ma za zadanie wygenerować w Google Analytics statystyki przedstawiające zdefiniowaną przez użytkownika ilość odwiedzeń strony w przeciągu podanego przedziału czasu. Statystyki mają obejmować aplikacje klienckie (user agent) i lokalizację użytkowników (w obrębie Polski). Aplikacja będzie całkowicie konsolowa --- nie będzie posiadać graficznego interfejsu. Jedynymi parametrami możliwymi do wyspecyfikowania przez użytkownika (i jednocześnie wymaganymi) są:
\begin{itemize}
\item Tracking ID --- identyfikator konta w Google Analytics
\item URL --- adres strony
\item ilość odwiedzeń
\item wielkość przedziału czasu (w s)
\end{itemize}

\subsection{Jakość}
Generowane przez aplikację statystyki dotyczące lokalizacji i aplikacji klienckich są realistyczne. Oznacza to, że rozkład generowanych lokalizacji jest zgodny z rozkładem gęstości zamieszkania na terenie Polski, a rozkład aplikacji klienckich --- z ich procentową popularnością w Polsce. Ponadto odwiedzenia generowane są liniowo w całym przedziale czasu, z pewną wariancją --- z tego powodu ostateczna ilość odwiedzeń może się nieznacznie różnić od oczekiwanej. Różnica ta nie przekroczy 1\%.

\subsection{Przykład użycia}
Jedyny możliwy sposób użycia aplikacji. Użytkownik uruchamia program z linii poleceń, przekazując mu wymagane argumenty. Po skończeniu działania programu wszystkie oczekiwane statystyki zostały poprawnie wysłane do Google Analytics.

\subsubsection{Aktorzy}
\begin{itemize}
\item Użytkownik
\item Google Analytics
\end{itemize}

\subsubsection{Dane wejściowe}
\begin{itemize}
\item Tracking ID
\item URL
\item liczba odwiedzeń (z góry ustalonego przedziału)
\item wielkość przedziału czasu (w s)
\end{itemize}

\subsubsection{Warunki wstępne}
\begin{itemize}
\item Działające połączenie z internetem
\item Posiadanie strony internetowej
\item Posiadanie poprawnie skonfigurowanego konta w Google Analytics
\end{itemize}

\subsubsection{Warunki końcowe}
Do serwera Google Analytics zostało wysłane dokładnie tyle odwiedzeń ile zostało przekazane jako parametr w podanym przedziale czasu.

\subsubsection{Rezultat}
Użytkownik posiada na swoim koncie Google Analytics tyle dodatkowych odwiedzeń ile chciał, uwzględniających lokalizację i przeglądarkę odwiedzających.

\subsubsection{Scenariusz główny}
\begin{enumerate}
\item Użytkownik uruchamia program z linii poleceń przekazując mu odpowiednie argumenty
\item Program weryfikuje poprawność podanych argumentów
\item Jeśli argumenty były poprawne program wysyła dane do Google Analytics, wypisuje informację o powodzeniu i kończy działanie. W przeciwnym przypadku program wypisuje odpowiedni komunikat o błędzie i kończy działanie.
\end{enumerate}

\subsubsection{Scenariusz neagtywny}
\begin{enumerate}
	\item Użytkownik jako argument podaje nierealną liczbę odwiedzeń w podanym przedziale czasu, niepoprawne Tracking ID\@. W takiej sytuacji aplikacja wypisuje na standardowe wyjście odpowiedni komunikat i oczekuje nowych danych.

\item Użytkownik podaje poprawne argumenty, jednak liczba odwiedzeń w stosunku do wielkości przedziału czasu jest na tyle duża, że aplikacja nie jest w stanie wygenerować wiarygodnych danych, generujących oczekiwane statystyki. Wiarygodność danych jest określana na podstawie przeciętnego ruchu w sieci na terenie Polski. W takiej sytuacj aplikacja wypisuje na standardowe wyjście odpowiednie ostrzeżenie, z pytaniem czy użytkownik chce zmodyfikować argumenty.

\item W trakcie działania zostaje utracone połączenie z internetem. Aplikacja daje użytkownikowi możliwość wyboru jednej z 3 opcji:
\begin{enumerate}
\item po uzyskaniu połączenia generowanie dalej wejść
\item po uzyskaniu połączenia rozpoczęcie ponownego generowania wejść
\item zaniechanie generowania.
\end{enumerate}

\end{enumerate}

\section{Wymagania niefunkcjonalne}

\subsection{Wydajność}
Aplikacja jest w stanie generować ponad tysiąc odwiedzeń na sekundę na maszynie z procesorem Intel Core i5--4570, 3.2GHz (8GB RAM) i łączem 20 MB/s.

\subsection{Bezpieczeństwo}
Aplikacja nie zapisuje żadnych danych użytkownika i wszystkie dane przesyłane są poprzez szyfrowane połączenie (HTTPS).

\end{document}
