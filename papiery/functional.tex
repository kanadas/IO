\documentclass{article}

    \usepackage[utf8]{inputenc}
    \usepackage{polski}
    \usepackage{titlesec}
    
    \title{Generator ruchu Google Analytics \\ Iteracja II specyfikacja wymagań}
    \author{Bartłomiej Dalak \and Bartłomiej Karwowski \and Bartosz Gromek \and Tomasz Kanas}
    
    \begin{document}
    \maketitle
    
    \section{Opis ogólny}
    \begin{itemize}
    \item Celem pierwszej iteracji jest napisanie skryptu generującego statystyki odwiedzania danej strony w podanym przez użytownika przedziale czasu w Google Analytics.
    \item Celem drugiej iteracji jest stworzenie systemu kolejkowania uruchomień skryptu o określonej godzinie, wraz z interfejsem użytkownika w postaci aplikacji webowej.
    \end{itemize}

    \vspace{7\baselineskip}

    \section{Wymagania funkcjonalne}
    
    Aplikacja ma umożliwić użytkownikowi wypełnienie formularza, którego efektem końcowym będzie przesłanie do Google Analytics oczekiwanych statystyk dotyczących odwiedzeń danej strony. Konieczne do tego jest posiadanie przez użytkownika poprawnie skonfigurowanego konta w Google Analytics. Aplikacja ma za zadanie utworzenie oraz edytowanie procesu generującego w Google Analytics statystyki przedstawiające zdefiniowaną przez użytkownika ilość odwiedzeń strony w przeciągu podanego przedziału czasu. Głównym panelem zarządzania aplikacji będzie prosty formularz do dodawania tasków reprezentujących pojedyńcze uruchomienie skryptu w określonym dniu i godzinie. Z jego pomocą, klikając odpowiednie przyciski zawarte w formularzu, użytkownik będzie w stanie dodawać nowe taski, a także edytować lub usuwać/anulować już istniejące. W danej chwili do każdego z tasków obecnych w aplikacji będzie przypisany jeden z siedmu możliwych stanów, informujących o jego statusie:
    
    \begin{itemize}
        \item New -- zadanie dodane do bazy
        \item Ready -- zadanie zapisane, czekające na wykonanie
        \item In progress -- zadanie w trakcie wykonywania
        \item Cancelled -- zadanie zatrzymane
        \item Deleted -- zadanie usunięte, zanim zaczęło się wykonywać
        \item Done -- zadanie zostało wykonane
        \item Error -- wystąpiły problemy w trakcie wysyłania
        \end{itemize}
    
    Naciśnięcie przycisku 'Add new' utworzy nowy task o statusie 'New'. Po wprowadzeniu danych można go zapisać, naciskając przycisk 'Save'. Użytkownik, wypełniając formularz, musi podać następujące dane:
    
    \begin{itemize}
    \item Tracking ID --- identyfikator konta w Google Analytics
    \item URL --- względny adres strony, tzn. adres podstrony (url'a przypisanego do Tracking ID), która ma mieć zarejestrowane wyświetlenia.
    \item Ilość odwiedzeń
    \item Wielkość przedziału czasu (w s)
    \item Datę i godzinę.
    \end{itemize}

    \vspace{1\baselineskip}
    
    \subsection{Jakość}
    Każdy dodany task będzie wysłany o konkretnym czasie, zdefiniowanym przez użytkownika. Aplikacja będzie pilnować podanych tasków i wysyłać statystyki do Google Analytics o odpowiedniej, podanej wcześniej godzinie. Aplikacja będzie wysyłać zapytania do usługi w przeciągu całego przedziału czasu podanego przez użytkownika. Ponadto odwiedzenia generowane są liniowo w całym przedziale czasu, z pewną wariancją --- z tego powodu ostateczna ilość odwiedzeń może się nieznacznie różnić od oczekiwanej. Różnica ta nie przekroczy 1\%.

    \vspace{2\baselineskip}
    
    \subsection{Przykład użycia}
    \begin{itemize}
    \item Dodanie nowego taska. Użytkownik wypełnia formularz na stronie, po czym klika przycisk 'add new'. W tabeli tasków pojawia się nowy wiersz zawierający
        id zlecenia, datę, liczbę wyświetleń, czas ich generowania oraz url strony dla której mają być przesłane. Jeśli skrypt wysyłający dane do GA nie zwróci żadnych błędów, to w momencie zakończenia generowania wyświetleń task zmieni status na 'Done', w przeciwnym wypadku jego status zmieni się na 'Error occured'.
    \item Edycja taska. Dopóki task ma status 'Ready', to użytkownik może edytować jego wszystkie właściwości (poza id taska). Edycja następuje poprzez modyfikację pól w tabeli i kliknięcie przycisku 'save'.
    \item Usunięcie taska. Przy każdym wierszu w tabeli, gdzie status jest 'Ready' albo 'In progress', aktywny jest przycisk w kształcie krzyżyka pozwalający na usunięcie/zatrzymanie działania taska (w zależności od jego statusu). Task ze statusu 'Ready' zmienia się na 'Deleted', 'In progress' na 'Canceled'.
    \end{itemize}
    
    \subsubsection{Aktorzy}
    \begin{itemize}
    \item Użytkownik
    \item Google Analytics
    \end{itemize}
    
    \subsubsection{Dane wejściowe}
    \begin{itemize}
    \item Tracking ID
    \item URL
    \item Liczba odwiedzeń (z góry ustalonego przedziału)
    \item Wielkość przedziału czasu (w s)
    \item Data i godzina uruchomienia skryptu.
    \end{itemize}
    
    \subsubsection{Warunki wstępne}
    \begin{itemize}
    \item Działające połączenie z internetem
    \item Posiadanie strony internetowej
    \item Posiadanie poprawnie skonfigurowanego konta w Google Analytics
    \end{itemize}
    
    \subsubsection{Warunki końcowe}
    Do serwera Google Analytics zostało wysłane dokładnie tyle odwiedzeń ile zostało przekazane jako parametr w podanym przedziale czasu.
    
    \subsubsection{Rezultat}
    Użytkownik posiada na swoim koncie Google Analytics tyle dodatkowych odwiedzeń ile chciał, uwzględniających lokalizację i przeglądarkę odwiedzających.
    
    \subsubsection{Scenariusz główny}
    \begin{enumerate}
    \item Użytkownik wchodzi na stronę, dodaje nowy task klikając przycisk 'Add new' lub edytuje już istniejący ze stanem 'Ready', po czym wypełnia formularz.

    \item Strona weryfikuje poprawność danych. Jeśli nie są poprawne to wypisywany jest stosowny komunikat.

    \item W przeciwnym przypadku w tabeli pojawia się nowy wiersz ze statusem 'Ready'. W danym przez użytkownika dniu i godzinie uruchamiany jest skrypt z odpowiednimi argumentami. Wtedy status  w tabeli zmienia się na 'In progress'. Skrypt wysyła dane do Google Analytics. W przypadku, gdy skrypt kończy działanie z powodzeniem, task przyjmuje ostatecznie status 'Done'. Jeśli wystąpi błąd to statusem końcowym jest 'Error Occured'.
    \end{enumerate}
    
    \subsubsection{Scenariusz negatywny}
    \begin{enumerate}

    \item Użytkownik nie wypełnia części/całego formularza. W takim przypadku aplikacja prosi jeszcze raz o wypełnienie wszystkich jego pól.
    
    \item Użytkownik wypełnia formularz nierealną liczbą odwiedzeń w ogóle albo w podanym przedziale czasu bądź niepoprawny Tracking ID\@. W takiej sytuacji aplikacja webowa wypisuje odpowiedni komunikat i oczekuje nowych danych.
    
    \item Użytkownik próbuje edytować task ze stanem 'Deleted', 'Cancelled' lub 'Done'. W takim przypadku aplikacja nie pozwala mu na to, wypisując odpowiedni komunikat.
    
    
    \end{enumerate}
    
    \section{Wymagania niefunkcjonalne}
    
    \subsection{Wydajność}
    Aplikacja jest w stanie generować ponad tysiąc odwiedzeń na sekundę na maszynie z procesorem Intel Core i5--4570, 3.2GHz (8GB RAM) i łączem 20 MB/s.
    
    \subsection{Bezpieczeństwo}
    Aplikacja nie zapisuje żadnych danych użytkownika i wszystkie dane przesyłane są poprzez szyfrowane połączenie (HTTPS).
    
    \end{document}
    